\documentclass[a4paper,10pt]{article}
\usepackage{listings}
\lstloadlanguages{PHP,HTML}
\usepackage{a4wide}
\usepackage{url}
\usepackage[utf8x]{inputenc}
\usepackage{times}
\usepackage[german]{babel}

\usepackage{ifpdf}
\ifpdf
\usepackage[pdftex,
            pagebackref=true,
            colorlinks=true,
            linkcolor=blue,
            unicode
           ]{hyperref}
\else
\usepackage[ps2pdf,
            pagebackref=true,
            colorlinks=true,
            linkcolor=blue,
            unicode
           ]{hyperref}
\usepackage{pspicture}
\fi

\title{TemplateEngine2}
\author{Obexer Christoph}
\date{18.05.2010}

\pdfinfo{
  /Title    (TemplateEngine2)
  /Author   (Obexer Christoph)
  /Creator  (Obexer Christoph)
  /Producer (Obexer Christoph)
  /Subject  (TemplateEngine2)
  /Keywords (Template Engine)
}

\begin{document}
\maketitle
\pagebreak

\tableofcontents
\pagebreak

\section{TemplateEngine2 Allgemein}

\subsection{TemplateEngine Version 1 Kompatibilit\"at}
Die syntax der TemplateEngine v1 wird vollst\"andig verstanden - mit einer einzigen Au\ss{}nahme:\newline
Die {\tt \{ELSE\}} Direktive muss nun als {\tt \{IF:ELSE\}} geschrieben werden, Grund daf\"ur ist
das das {\tt \{ELSE\}} als normale Template Variable verstanden werden kann und in Zukunft nicht mehr
eindeutig sein wird(inline FOREACH).\newline
Die methode {\tt getTemplatePath} gibt set Version 2.0 nur noch den TemplatePath Anteil zur\"uck, und nicht mehr
RootPath / TemplatePath!

\subsection{Features}
\begin{itemize}
	\item Plugin Interface - alles wird durch Plugins erledigt!
	\item Escape Method Interface - Plugins k\"onnen auf Escape Methoden zur\"uckgreifen um die Ausgabe zu filtern,...
	\item Ein paar Methoden um spezielle Aufgaben zu erleichtern(User Messages, addCSS, addJS, setPageTitle,...)
	\item Debugging Funktionen um Debug-Informationen in den Browser zu bekommen.
	\item Da alles als Plugin realisiert ist kann auch alles deaktiviert werden - auch die Skalar Ersetzung :)
\end{itemize}


\section{TemplateEngine2 API}

\subsection{User API}

\subsubsection{Inst()}
Die Inst Methode gibt ein Objekt der TemplateEngine zur\"uck, das kann verwendet werden um die Tipparbeit zu verringern.

\subsubsection{clear()}
Mit der {\tt clear} Methode werden alle gesetzten Variablen gel\"oscht und automatische Variablen wieder zur\"uckgesetzt.

\subsubsection{setTemplatePath(\$path)}
Setzt den Pfad in dem nach Templates gesucht werden soll. Relativ zum RootPath und zum Browser.
Der TemplatePath steht Templates als {\tt \{TEMPLATE\_PATH\}} zur Verf\"ugung.

\subsubsection{getTemplatePath()}
Gibt den zuvor mit setTemplatePath gesetzten Teil zur\"uck, die Version 1.0 hat hier den Pfad mit dem RootPath Anteil geliefert!

\subsubsection{setRootPath(\$path)}
Setzt den Pfad der dem Browser als RootPath zur Applikation dient, diese Variable ist als {\tt \{ROOT\_PATH\}} in den Templates verf\"ugbar.

\subsubsection{getRootPath()}
Gibt den zuvor mit {\tt setRootPath} gesetzten Pfad zur\"uck.

\subsubsection{output(\$basetemplate, \$havingSession)}
Verarbeitet das Template mit dem in {\tt \$basetemplate} \"ubergebenen Namen (gesucht wird es im TemplatePath) und schickt das Ergebnis zum Browser.
Der {\tt \$havingSession} Parameter gibt an ob vor der Verarbeitung nur {\tt TE\_static\_setup} oder auch {\tt TE\_setup} aufgerufen werden soll.\newline
{\tt \$havingSession} = {\tt true} -{\textgreater} {\tt TE\_static\_setup}, {\tt TE\_setup} aufrufen.\newline
{\tt \$havingSession} = {\tt false} -{\textgreater} {\tt TE\_static\_setup} aufrufen.

\subsubsection{processTemplate(\$basetemplate, \$havingSession)}
Verarbeitet das Template mit dem in {\tt \$basetemplate} \"ubergebenen Namen (gesucht wird es im TemplatePath) und gibt den resultierenden Content
als String zur\"uck. Der {\tt \$havingSession} Parameter gibt an ob vor der Verarbeitung nur {\tt TE\_static\_setup} oder auch {\tt TE\_setup} aufgerufen werden soll.\newline
{\tt \$havingSession} = {\tt true} -{\textgreater} {\tt TE\_static\_setup}, {\tt TE\_setup} aufrufen.\newline
{\tt \$havingSession} = {\tt false} -{\textgreater} {\tt TE\_static\_setup} aufrufen.

\subsubsection{set(\$name, \$value)}
Setzt die Variable mit dem Namen {\tt \$name} auf den Wert {\tt \$value}, die Eingebauten Direktiven verstehen hier nur Skalare und einfache Arrays.
Die Erweiterbarkeit der TemplateEngine2 macht es hier allerdings auch m\"oglich beliebige Datenstrukturen zu verwalten und zu verwenden.

\subsubsection{get(\$name, \$default = null)}
Ermittelt den Wert einer Variablen, sollte der Wert noch nicht bekannt sein wird der wert des 2. Parameters({\tt \$default}) zur\"uckgegeben.
Diese Methode arbeitet nur auf Basis des globalen Kontext und sollte deswegen nicht in einem Template-Plugin verwendet werden, daf\"ur wurde {\tt lookupVar} implementiert.

\subsubsection{delete(\$name)}
L\"oscht den wert der f\"ur die Variable {\tt \$name} gespeichert wurde.

\subsubsection{Error(\$error)}
Diese Methode f\"ugt dem standardm\"a\ss{}ig vorhandenem Array {\tt TE\_ERRORS} eine Fehlermeldung hinzu. Diese Fehlermeldungen sind dazu gedacht dem User angezeigt zu werden.

\subsubsection{Warning(\$warning)}
Diese Methode f\"ugt dem standardm\"a\ss{}ig vorhandenem Array {\tt TE\_WARNINGS} eine Warnmeldung hinzu. Diese Warnmeldungen sind dazu gedacht dem User angezeigt zu werden.

\subsubsection{Info(\$info)}
Diese Methode f\"ugt dem standardm\"a\ss{}ig vorhandenem Array {\tt TE\_INFOS} eine Infomeldung hinzu. Diese Infomeldungen sind dazu gedacht dem User angezeigt zu werden.

\subsubsection{setTitle(\$title)}
Diese Methode setzt den Titel der Seite (genauer gesagt die Variable {\tt PAGE\_TITLE}).

\subsubsection{header(\$html)}
Mit {\tt header} wird der \"ubergebene HTML-Code an den HTML-Head angeh\"angt.

\subsubsection{addCSS(\$css)}
F\"ugt dem HTML-Head ein {\tt link} Tag hinzu der als {\tt href} den \"ubergebenen String hat.

\subsubsection{addJS(\$js)}
F\"ugt dem HTML-Head ein {\tt script type='text/javascript'} Tag hinzu der als {\tt src}
 den \"ubergebenen String hat.

\subsubsection{setFileDebugMode(\$mode)}
Setzt man {\tt \$mode} auf {\tt true} so wird jeder geladenen Datei mit HTML-Kommentaren
der Dateiname vorne und hinten an den Inhalt angeh\"angt.

\subsubsection{setForceTplExtension(\$mode)}
Setzt man {\tt \$mode} auf {\tt true} so wird jede Datei, die nicht {\tt .tpl} als Dateityp
 hat zur\"uckgewiesen.

\subsubsection{setJailToTemplatePath(\$mode)}
Setzt man {\tt \$mode} auf {\tt true} so wird jede Datei, die nicht im {\tt TEMPLATE\_PATH}
 gefunden wird zur\"uckgewiesen.

\subsection{Plugin API}

\subsubsection{pushContext(\$templateString, array \$context)}
Damit wird ein neuer Context auf den Context-Stack geschoben und verarbeitet.
{\tt \$templateString} ist der zu verarbeitende Template String, das Array {\tt \$context} anth\"alt alle im Kontext verf\"ugbaren Variablen.

\subsubsection{escape(\$escaper, \$value)}
Mit dieser Methode wird die Escape methode mit dem in {\tt \$escaper} \"ubergebenen Namen ausgef\"uhrt und das Ergebnis des escapens von {\tt \$value} zur\"uckgegeben.

\subsubsection{registerPlugin(\$plugin, \$regexp, \$callback)}
Diese Methode registriert ein Plugin f\"ur die Verwendung in Templates.
\begin{itemize}
  \item[\texttt{\$name}] der Name des Plugins - muss eindeutig sein, ansonsten wird das zuvor registrierte Plugin \"uberschrieben. Namen mit {\tt TE\_} am Anfang sind reserviert!
  \item[\texttt{\$regexp}] der Regul\"are Ausdruck mit dem nach Vorkommen f\"ur dieses Plugin gesucht wird.
  \item[\texttt{\$callback}] die Funktion die mit Treffern f\"ur das Plugin aufgerufen wird - bei jedem Aufruf wird nur ein Treffer verarbeitet.
\end{itemize}

{\bf Das Callback Interface:}\newline
Die Callback Funktion erh\"alt zwei Parameter:
\begin{itemize}
  \item[{\tt \$context}] ein Array in dem alle Werte des Aktuellen Kontext entahlten sind.
    Wird der angeforderte Wert im aktuellen Context nicht gefunden, dann kann {\tt lookupVar} verwendet werden um im Context-Stack nach Unten zu suchen.
  \item[{\tt \$match}] der Parameter der von {\tt preg\_replace\_callback} an die Interne Callback Funktion \"ubergeben wurde,
    sie enth\"alt alle Matches des Regul\"aren Ausdrucks - so wie er registriert wurde.
\end{itemize}
Der return-Wert der Funktion muss entweder ein String oder ein Boolean({\tt false}) sein. Mit {\tt false} lehnt das Plugin den Treffer ab und er wird unver\"andert belassen, ansonsten wird der Treffer mit dem zur\"uckgegebenen String ersetzt.

\subsubsection{unregisterPlugin(\$plugin)}
Hebt die Registrierung eines Plugins auf - es kann anschlie\ss{}end nicht mehr verwendet werden.

\subsubsection{lookupVar(\$name, \&\$value)}
{\tt lookupVar} findet den Wert der in {\tt \$name} \"ubergebenen Variable und gibt {\tt true} zur\"uck wenn der Wert
auch gefunden wurde, ansonsten wird {\tt false} zur\"uchgegeben und der Wert von {\tt \$value} bleibt unver\"andert.
Diese Methode sollte nur dann aufgerufen werden wenn die Variable im Kontext der dem Plugin Callback \"ubergeben wurde nicht enthalten ist.

\subsubsection{getFile(\$name, \&\$content)}
{\tt getFile} l\"adt eine Datei und gibt bei Erfolg (return {\tt true}) in ihrem 2. Parameter den Inhalt zur\"uck. Diese Funktion beachtet
dabei {\tt \$force\_tpl\_extension}, {\tt \$jail\_to\_template\_path} und {\tt \$debug\_files}.

\subsection{Escape Method API}

\subsubsection{registerEscapeMethod(\$method, \$callback)}
Mit dieser Methode wird die Escape Methode mit dem Namen {\tt \$method} f\"ur die Verwendung registriert.\newline
\newline
{\bf Das Callback Interface:}\newline
Die Callback Funktion erh\"alt 2 Parameter:
\begin{itemize}
  \item[{\tt \$value}] der Rohwert der f\"ur die zu escapende Variable gesetzt wurde - was das ist h\"angt au\ss{}schlie\ss{}lich davon ab was f\"ur die Variable gesetzt wurde - beispielsweise ein String, ein Boolean, ein Objekt oder ein Array,...
  \item[{\tt \$config}] wurde f\"ur die Escape Methode eine Konfiguration(beispielsweise eine Zeitzohne, oder Sprache,...) mit {\tt setEscapeMethodConfig} hinterlegt wird diese hier \"ubergeben, andernfalls ist dieser Parameter {\tt null}.
\end{itemize}
Der return Wert der Funktion sollte ein String sein. Der Wert wird dann zum ausl\"oser der Escape methode gegeben.

\subsubsection{unregisterEscapeMethod(\$method)}
L\"oscht eine registrierte Escape Methode wieder.

\subsubsection{setEscapeMethodConfig(\$method, \$config)}
Kann beliebige Daten als Konfiguration f\"ur eine Escape Methode speichern.

\subsubsection{getEscapeMethodConfig(\$method}
Liefert die aktuell gespeicherte Konfiguration f\"ur die Escape Methode.

\subsection{Misc API}

\subsubsection{LogMsg(\$msg, \$success = true, \$mode = TEMode :: debug, \$finished= true)}
Mit der LogMsg Methode kann Debug Information aufgezeichnet werden, die Informationen werden nur aufgezeichnet, wenn
der \"ubergebene {\tt \$mode} gr\"o\ss{}er oder gleich dem Eingestellten ist. Mit {\tt \$finished = false} wird die Nachricht
erst mit dem n\"achsten Aufruf({\tt \$finished = true}) an den Puffer angeh\"angt, dieser kann auch ein neues Log-Level f\"ur
die gesamte Nachricht angeben. {\tt \$success} gibt an ob der Vorgang erfolgreich war.

\subsubsection{captureTime(\$milestone)}
{\tt captureTime} kann dazu verwendet werden Timing-Probleme beim Aufbau einer Seite auf die Schliche zu kommen.
Der {\tt \$milestone} Parameter ist einfach ein Name f\"ur die Position im Ablauf des Seitenaufbaus,
beispielsweise {\tt startTE} f\"ur den Start der Template Verarbeitung oder {\tt stopTE} f\"ur das Ende der
Template Verarbeitung. {\tt printTimingStatistics} wird automatisch bei scriptbeendung aufgerufen und gibt die aufgezeicheten
Milestones aus und markiert damit das Ende der Skriptausf\"uhrung (da {\tt TEincluded} den Anfang markiert kann aus
dem Time Offset von {\tt printTimingStatistics} die Skriptlaufzeit abgelesen werden).

\section{TemplateEngine Plugins}

\subsection{Skalar Ersetzung}
Die Skalar Ersetzung ersetzt alle Vorkommen von {\tt \{VAR\}} mit dem Wert der der TemplateEngine f\"ur {\tt VAR} bekannt
gemacht wurde(siehe {\tt set}). Die Skalar Ersetzung verwendet auch Escape Methoden, so kann beispielsweise die Anzahl
der Elemente eines Arrays names {\tt ARMIES} mit {\tt \{ARMIES|LEN\}} Ausgegeben werden. Vorraussetzung f\"ur die
Verwendung von Escape Methoden ist das diese Registriert sind - gilt auch f\"ur {\tt LEN}.

\subsection{LOAD}
Mit der {\tt LOAD} Direktive wird an der Stelle des Vorkommens ein anderes Template geladen.
Verwendung: {\tt \{LOAD=path/to/file.tpl\}} der Pfad ist relativ zum {\bf TemplatePath}.

\subsection{LOAD\_WITHID}
Mit der {\tt LOAD\_WITHID} Direktive wird an der Stelle des Vorkommens ein anderes Template geladen.
zus\"atzlich zur normalen {\tt LOAD} wird innerhalb der geladenen Datei {\tt \{LOAD:ID\}} durch den angegebenen Wert ersetzt.
Verwendung: {\tt \{LOAD\_WITHID=path/to/file.tpl;the-new-id\}} der Pfad ist relativ zum {\bf TemplatePath}.

\subsection{LOGLEVEL}
Setzt das Log-Level der TemplateEngine2 dynamisch auf den gegebenen Wert. Verwendung: {\tt \{LOGLEVEL=ERROR\}}.
Es sind folgende Werte erlaubt:
\begin{itemize}
 \item DEBUG
 \item WARNING
 \item ERROR
 \item NONE
\end{itemize}

\subsection{FOREACH}
{\tt FOREACH} wird dazu verwendet Arrays mit einfachem Aufbau darzustellen,
gegeben sei ein Array mit dem Namen {\tt USERS} mit folgendem Aufbau:\newline
[[NAME=Mea, LEVEL=admin], [NAME=Schalk, LEVEL=admin], [NAME=cobexer, LEVEL=admin]]\newline
bei der Ausf\"uhrung wird f\"ur jedes Element im Array ein eingener Context erzeugt, das hat einerseits
einen gewissen Geschwindigkeitsvorteil andererseits aber auch den Vorteil, da\ss{} alle Plugins in diesem
Kontext ganz normal arbeiten k\"onnen, weitere {\tt FOREACH} oder {\tt IF} oder jedes andere Plugin.\newline
Zusa\"tzlich ist im neuen Context eine variable namens {\tt \{ODDROW\}} verf\"ugbar die entweder {\tt odd} oder leer ist
je nachdem ob die aktuelle Zeile ungerade ist oder nicht.\newline
Der Index der {\tt foreach} Iteration wir mit dem speziellen Token {\tt \{FOREACH:INDEX\}} bereitgestellt.\newline
Verwendung:
\lstset{language=HTML}
\begin{lstlisting}
<ul>
{FOREACH[USERS]=userlist.tpl}
</ul>
\end{lstlisting}
Inhalt von \texttt{userlist.tpl}:
\begin{lstlisting}
<li>{NAME} ({LEVEL})</li>
\end{lstlisting}
Ergebnis:
\begin{lstlisting}
<ul>
<li>Mea (admin)</li>
<li>Schalk (admin)</li>
<li>cobexer (admin)</li>
</ul>
\end{lstlisting}


\subsection{SELECT}
Die {\tt SELECT} Direktive dient dazu HTML-Select Elemente zu generieren. Verwendung:\newline
\begin{lstlisting}
<select name="myselect">
{SELECT=MYOPTIONS}
</select>
\end{lstlisting}
Wobei {\tt MYOPTIONS} ein Array mit {\tt NAME} / {\tt VALUE} Paaren ist:\newline
[[NAME=Mea, VALUE=1], [NAME=Schalk, VALUE=2]]\newline
Ergebnis:
\begin{lstlisting}
<select name="myselect">
<option value="1">Mea</option>
<option value="2">Schalk</option>
</select>
\end{lstlisting}

\subsection{IF}
Die{\tt IF} Direktive wird dazu verwendet eine Bedingung auszuwerten.
Aufbau der {\tt IF} Direktive:\newline
{\tt
\{IF(VAR\_NAME[|ESCAPE\_METHOD] operator wert)\}}\newline
 ist die Bedingung wahr bleibt dieser Teil erhalten\newline
 {\tt \{IF:ELSE\}}\newline
 ansonsten dieser\newline
{\tt \{/IF\}}\newline
f\"ur den Operator kan einer der Folgenden verwendet werden:
\begin{itemize}
  \item gt (\textgreater)
  \item lt (\textless)
  \item eq (==)
  \item gte (\textgreater=)
  \item lte (\textless=)
  \item ne (!=)
\end{itemize}
Der Wert {\tt 'null'} hat die besondere Bedeutung von {\tt null} in PHP,
er kann mit {\tt ne} dazu verwendet werden zu pr\"ufen ob eine Variable gesetzt ist.

\section{TemplateEngine Escape Methods}
Escape Methoden werde dazu verwendet Template Variablen zu ver\"andern, beispielsweise
um zu verhindern das User HTML- oder Template-Code einschleusen k\"onnen.

\subsection{LEN}
Die Escape Methode {\tt LEN} kann nur auf Strings und Arrays angewendet werden, das
Ergebnis der Escape Operation ist entweder die Stringl\"ange oder die Anzahl der
Elemente im Array.\newline
Anwendungsbeispiel:\newline
{\tt Anzahl der Spieler: \{USER\_ARRAY|LEN\}}

\section{spezielle URL Parameter}

\begin{itemize}
  \item[force\_debug] setzt den TemplateEngine Mode auf Debug, es werden alle {\tt \{LOGLEVEL=\dots \}} aufrufe ignoriert.
  \item[show\_timing] listet alle "`Milestones"' die mit {\tt captureTime} aufgezeichent wurden auf.
  \item[debug\_files] h\"angt allen eingebundenen Dateien einen HTML-Kommentar mit dem Dateinamen vorne und hinten an.
  \item[no\_inline] sucht nach allen inline {\tt style} attributen und entfernt sie.
  \item[te\_dump] gibt nach dem verarbeiten der Seite alle Template-Variablen und deren Werte aus.
\end{itemize}


% GomBG specific extensions:

\section{GomBG spezifische Features}
F\"ur GomBG sind einige extra Plugins vorhanden:

\subsection{TemplateEngine Plugins}

\subsubsection{LINKTO}
Mit {\tt LINKTO} werden HTML-Links zu verschiedenen 'Objekten' in GomBG erstellt.\newline
{\tt \{LINKTO=WHAT;TYPE=TP;TEXT=TX;TITLE=TI\}}\newline
Wobei man {\tt WHAT, TP, TX} und {\tt TI} entsprechend ersetzen muss.
\begin{itemize}
	\item[LINKTO=] der einzusetzende Wert ist beispielsweise der Nick, Allianzname, Uid, Pos,\dots
	\item[TYPE=] Der {\tt TYPE} ist zur Zeit einer der folgenden 7 Werte:
	\begin{itemize}
		\item[PLAYER] (Nick) Link zum Profil.
		\item[SENDMSG] (Nick) Link zum Nachrichtenmodul mit dem Nick als Empf\"anger.
		\item[ALLI] (Allianzname) Link zur Allianzseite.
		\item[MAP] (Pos) Link zum Feld auf der Map ([\textbackslash d\{5\}] im {\tt TEXT} wird durch [ XX / YY / Z ] ersetzt).
		\item[ARMY] (Army ID) Link zum bearbeiten einer Armee.
		\item[BUILDING] (Building ID) Link zur Beschreibungsseite eines Geb\"audes mit allen Stufen, Abh\"angigkeiten und so(gibt es noch nicht).
		\item[FAV] (Nick) Link zum NAchrichtenmodul um einen User als Buddy zu speichern.
	\end{itemize}
	\item[TEXT=] Dieser Text wird dem User angezeigt({\textless}a \dots{\textgreater}TEXT{\textless}/a{\textgreater})
	\item[TITLE=] Das ist der Titel der in einem Tool-Tip angezeigt wird wenn man mit dem Cursor auf einen Link zeigt.
\end{itemize}

\subsection{TemplateEngine Escape Methods}

\subsubsection{TIMESTAMP}
Die {\tt TIMESTAMP} escape Methode formatiert einen als Integer gesetzten Datumswert in dem default Format f\"ur User.
Beispiel:\newline
{\tt \{FINISHED\_AT|TIMESTAMP\}}

\subsection{spezielle URL Parameter}

\begin{itemize}
  \item[show\_queries] zeigt eine Liste mit allen Datenbak Queries, mehrfach ausgef\"uhrten Queries und der Ausf\"uhrungszeit, das Passwort ist am ende der {\tt inc/DB.php} zu finden.
\end{itemize}


\end{document}
